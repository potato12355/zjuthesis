\cleardoublestylepage{common}

\section{Overleaf 使用注意事项}

如果你在Overleaf上编译本模板,请注意如下事项:

\begin{itemize}
    \item 删除根目录的 ``.latexmkrc'' 文件,否则编译失败且不报任何错误
    \item 字体有版权所以本模板不能附带字体,请务必手动上传字体文件,并在各个专业模板下手动指定字体。
        具体方法参照 GitHub 主页的说明。
        {\bfseries 本模板的 Overleaf 版删除了仿宋字体设置,否则会编译超时}。
    \item 当前(2019年9月2日)的Overleaf使用TexLive 2017进行编译,但一些伪粗体复制乱码的问题需要TexLive 2019版本来解决。
        所以各位同学可以在Overleaf上编写论文,但务必使用本地的TexLive 2019来进行最终编译,以免产生查重相关问题。
        具体说明参照 GitHub 主页。
\end{itemize}

\subsection{节标题}

\subsubsection{小节标题}

\par 我们可以用includegraphics来插入现有的jpg等格式的图片,如\autoref{fig:zju-logo}。

\begin{figure}[ht]
    \centering
    \includegraphics[width=.4\linewidth]{logo/zju}
    \caption{\label{fig:zju-logo}浙江大学LOGO}
\end{figure}

\par 如\autoref{tab:sample}所示,这是一张自动调节列宽的表格。

\begin{table}[ht]
    \caption{\label{tab:sample}自动调节列宽的表格}
    \begin{tabularx}{\linewidth}{c|X<{\centering}}
        \hline
        第一列 & 第二列 \\ \hline
        xxx & xxx \\ \hline
        xxx & xxx \\ \hline
        xxx & xxx \\ \hline
    \end{tabularx}
\end{table}

\par 如\autoref{equ:sample},这是一个公式

\begin{equation}
    \label{equ:sample}
    A=\overbrace{(a+b+c)+\underbrace{i(d+e+f)}_{\text{虚数}}}^{\text{复数}}
\end{equation}

\par 如\autoref{code:sample}所示,这是一段代码。
计算机学院的代码样式可能与其他专业不同,
如有需要,可以从计算机学院专业模板中复制相关的代码样式设定。

\begin{lstlisting}[%
    language={C},
    caption={simple.c},
    label={code:sample}
]
#include <stdio.h>

int main(int argc, char *argv[])
{
    printf("Hello, zjuthesis\n");
    return 0;
}
\end{lstlisting}